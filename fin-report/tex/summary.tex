\chapter{结论}
  \label{chap:结论}
    观点和见解,归纳和总结
    本次商城的开发大体上完成了目标,尤其在样式的设计上投入了很多的精力,最终的样式效果与设计图几乎相符。
    本次项目的开发

\section{特色与意义}
  \label{sec:特色与意义}
    公众号运营,无需下载 App,直接关注公众号即可
    定制的商城,自己独立开发的一套框架,没有借用
    产品单一,
    技术新,此次的项目采用了强大的 AngularJS 框架开发,设计的应用代码量比较少,
    个人意义
    领域意义

\section{改进和完善}
  \label{sec:改进和完善}
    本次项目

\section{收获与感触}
  \label{sec:收获与感触}
  开发大型工程所注意的问题:多测试,模块化,熟练调试工具的使用
  学习方法:既要学又要练
  教训:理解原理很重要,
  大量学习
  框架与基础的思考:基础、原理很重要,学框架一定多探索背后的原理
    基础知识每天学,且时常复习
    每遇到遇到基础知识的问题,都要停下来测试
