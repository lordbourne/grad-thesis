% chapter: 绪论

\chapter{绪论}
  \label{chap:绪论}
    \section{课题背景与现状}
      \label{sec:课题背景与现状}
        \subsection{互联网+}
          \label{subsec:互联网_}
            2015 年 3 月 5 日,李克强总理在十二届人大三次会议上提出了“互联网+”的概念。“互联网+”的实质就是:互联网+传统行业。越来越多的企业与互联网结合,开辟了全新的经营模式————线上线下模式,线上接单,线下服务。如今这种商业模式已经深入到我们生活的方方面面,大量的企业开发了自己的商城网站,也开发了移动端的 App, 让我们的生活更加便捷,出行有滴滴,携程,购物有淘宝天猫,娱乐有美团等,这些应用让我们可以不受时间和空间的约束,去享受企业的便利服务。

        \subsection{微信平台}
          \label{subsec:微信平台}
            微信平台的出现为企业发展带来了新的机遇。微信平台有着超过 4 亿的用户基础,如此大的用户量蕴藏着巨大的价值。微信的价值主要在以下两点:
            \begin{itemize}
              \item 传播迅速
              \item 平台接口
            \end{itemize}
            \par
            传播迅速。我们知道,一副世界地图只需要 4 中颜色就可以表示出世界上几百个的国家,而不会产生混淆,这就是著名的四色理论。同样,在人脉的研究中也有个著名的理论——六度人脉理论,即一个人和世界上任何一个陌生人之间最多不超过 6 个陌生人,因此,如果一个企业把产品与服务做好,在自己的朋友圈取得信任后,就可以在圈内迅速传播,短时间内就可以打造企业的知名度。
            \par
            平台接口。传统的 App 必须要让客户去下载安装,而微信平台提供的网页接口可以让用户无需下载 App,用户关注公众号后就可以通过链接或菜单进入商城,直接下单购买,甚至省去了注册的步骤,因为微信平台可以获取用户的信息从而自动完成注册。

        \subsection{医疗大数据}
          \label{subsec:医疗大数据}
            而最近几年,健康医疗行业也迅速发展起来,2014 年,马云又投资10亿,拿下全国400家医院,覆盖90\% 的省份,我们可以看到其对健康医疗行业的重视。人们也在更多的关注健康与保健。而随着互联网的兴起的大数据行业也在迅速发展。海量的数据蕴含着丰富的信息,依据数据理论,通过一定的算法,就可以从数据中挖掘有价值信息。
            对健康状况的预测是大数据与医疗行业的一个重要的结合点。通过的先进的医疗器械获取大量患者的身体的各项指标,然后基于以往海量的样本进行数据建模,利用分类算法,就可以对一个人的健康状况作出比较精准的评估,进而对患者进行健康指导。

    \section{理论与技术}
      \label{sec:理论与技术}
        最近移动应用炒的最火的就是 H5 技术。H5 从狭义上说就是 HTML5,即最新版的 HTML 技术,从广义上说则是包含了 HTML5, CSS3, ES5甚至ES6在内的前端技术及框架,这些技术和框架可以让开发者迅速开发出华丽的移动应用。
        \par
        H5 相对于 H4 增加了很多新的特性,例如更多语义标签、canvas 绘图、多媒体标签、更强大的本地存储等;CSS 3 相对于上个版本功能更加多样化,弹性盒子模型、CSS 动画等;ES 指的是 JavaScript 的标准版本,最新的 ES6 相对于 ES 5 也有了更多扩展,如块作用域、of 语法、generator 函数等,不过还未大量采用,但一些最新的框架已经采用 ES6 开发了。这些只是底层的技术,要实现复杂的应用需要稳定成熟的框架,框架的出现于发展让应用的开发更为便捷与迅速。

    \section{项目特色与意义}
      \label{sec:项目特色与意义}
        我们的商城采用最新的 H5 技术,旨在打造一个特色的商城:定制的页面,专一的产品,服务与产品结合等,通过公众号运营,而不是独立的 App。
        此应用可以同时为客户提供服务与产品的信息,主要有两大门店:专注服务的门店与专注产品的门店,服务主要是基于艾灸的保健服务,产品主要是基于艾草的各种产品,如:艾绒棒、艾草保健品等,当然后期的服务与产品中还会继续扩展。应用同时设有艾灸百科专栏,让用户更多的了解艾灸文化,最后,应用还会专门记录客户的保健记录,重点不在于消费记录,而是健康状况记录,根据客户每次保健采集的指标,通过后台的大数据分析,评估健康状况,不过这些是以后的扩展功能。
        \par
        此项目的开发对我来说是一次难得的锻炼机会,让我在技术上有了很大提升。而且,此项目与我的职业道路紧密相关,我选择的职业就是前端开发,通过做项目,让我在公司入职之前给自己提前做了培训,为公司的入职做好准备。

    \section{发展趋势}
      \label{sec:发展趋势}
        个人认为,移动应用的发展有如下的趋势:
        \begin{itemize}
          \item 更多采用公众平台甚至小程序
          \item 专一化与定制化
        \end{itemize}
        \par
        上面提到,目前的移动开发越来越倾向于在微信平台开发,利用微信平台提供的巨大优势,可以让企业更快地推广、传播。现在有一种理论认为,企业的发展越来越向工匠化发展,即企业的价值更多地在于把产品做精,在精而不在多,这样产品就有了更多的品牌价值与权威性,做就做到最好,同时,为此开发的移动应用的就会越来越针对某种产品专门开发,这样的应用非常专一,而且轻量级。这样的应用还有一个特色,就是更强的定制性,有自己的创意,权威与特色这两大要素结合,让企业在未来竞争中更容易生存。



