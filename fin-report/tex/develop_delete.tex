
        \subsubsection{地理定位}
          \label{subsubsec:地理定位功能}
            首次进入商城,(todo)微信接口?还是自己写?应用会自动调用地理定位接口获取客户客户当前的位置,精确到县级,如:成华区,如果定位失败,则返回错误信息,位置自动设置为上一次的位置,并且提醒客户手动定位,点击左上方的按钮
            figtodo: 圈住左上方地理位置选择的图
            就会进入城市选择页,如下图所示:
            figtodo: 城市选择页
            然后就可以选择相应的城市或县区,为了更好地用户体验,如图,还设置了热门城市、最近选择的栏目。

        \subsubsection{列表展现}
          \label{subsubsec:列表展现}
            自动定位(或手动定位)好当前位置后,服务器会返回当前城市的保健店的列表,如图 todo 首页的设计图,简要地展示每个保健店的基本信息,如:店名、缩略图、艾灸床保健价格、其他服务与价格、详细地址以及右上角黄色的地理定位按钮,点击后会应用会打开地图接口,提供更好地体验,如下图:
            figtodo: 店的地图接口
            还有,如果一个店曾经预定过,则会以黄色的按钮提示已预订,点击预定按钮后就会进入店的详情页,展示店的详细信息以及进行预定操作。

        \subsection{商城搜索 todo: 待实现}
          \label{subsec:商城搜索_todo_待实现}
            如图figtodo: 首页设计图,还提供了搜索框,可以根据关键词搜索你想要的店,可以看到搜索框旁边没有搜索按钮,因为搜索是实时显示的,称为增量搜索,而且还采用了自动补全技术,根据最近的搜索关键词和列表加载时自动生成的关键词数据库进行模糊匹配,效果如下:设计非常人性化。
            figtodo: 搜索功能展示

    \subsection{地理定位功能 todo}
      \label{subsec:地理定位功能_todo}

    \subsection{搜索功能 todo}
      \label{subsec:搜索功能_todo}
