% 项目用到的技术与工具简介

\documentclass[UTF8]{ctexbook}

\ctexset{
    part/number = \chinese{part}% 用于解决 part 的标号不显示问题
}
\usepackage{hyperref}% 超链接
\hypersetup{
    colorlinks=false,% 去掉超链接颜色
    pdfborder=0 0 0% 取消超链接的边框
}
\usepackage{graphicx}% 图片管理
\graphicspath{{images/}}% 设置图片搜索路径
\usepackage{float,varwidth}% 浮动体
\usepackage{booktabs}% 三线表
\usepackage{tabularx}% 让表格自适应宽度与自动换行
\newcolumntype{Y}{>{\centering\arraybackslash}X}% 定义自适应列的居中格式 Y, 用 X 为左对齐(自适应列)
\usepackage{fancyhdr}% 页眉设置
\usepackage{xcolor}% 颜色宏包
\usepackage{listings}% 代码高亮
\definecolor{codegreen}{rgb}{0,0.6,0}
\definecolor{codegray}{rgb}{0.5,0.5,0.5}
\definecolor{codepurple}{rgb}{0.58,0,0.82}
\definecolor{backcolour}{rgb}{0.95,0.95,0.92}
\lstset{
    commentstyle=\color{codegreen},
    keywordstyle=\color{magenta},
    stringstyle=\color{codepurple},
    basicstyle=\footnotesize,% 代码字体大小
    breakatwhitespace=false,% 是否只在空白字符处断行
    breaklines=true,% 自动断行
    captionpos=b,% 标题位置为 bottom
    keepspaces=true,
    numbers=left,% 行号的位置
    numbersep=5pt,% 行号与代码的距离
    numberstyle=\tiny\color{codegray},% 行号样式
    stepnumber=2,% 隔行显示行号
    showspaces=false,
    showstringspaces=false,
    showtabs=false,
    tabsize=2
}

\begin{document}
  \chapter{技术与工具}
    \label{chap:技术与工具}
    本章主要介绍了项目所用到的技术和工具,技术包括:
    \begin{itemize}
      \item HTTP
      \item JSON 与 XML
      \item Apache 服务器的配置
      \item Nodejs 与 npm
      \item MongoDB
      \item H5
      \item CSS3
      \item Less
      \item ES5,ES6
      \item jquery
      \item MVC
      \item AngularJS
    \end{itemize}

    工具包括:

    \begin{itemize}
      \item Sublime Text3
      \item Chrome
      \item 微信 Web 开发者工具
      \item Git
    \end{itemize}

    \section{HTTP}
      \label{sec:http}
      \subsection{原理}
        \label{subsec:原理}


      什么是 HTTP

    \section{小结}
      \label{sec:小结}






\end{document}
