% 英文资料译文

\thesistranslationchinese
\section{深入理解ES6: Generators}
  我的深入理解 ES6 系列文章简要讨论了 JavaScript 语言第六版————ESMAScript 6(缩写为 ES6)的新特性。

  今天写的文章让我很兴奋,因为我们将去讨论 ES6 最神奇的一些特性。

  我说的神奇是什么意思?对于初学者,这些特性和以往的 JS 有很大的不同,你第一次接触会感到很难理解。在某种程度上,这些特性完全颠覆了语言的行为!

  不仅仅是这些,这些强大的特性可以简化代码、让你看清可怕的"回调函数地狱"的界限。

  我的溢美之词是不是太多了?那我们开始吧。

  \subsection{ES6 Generators}
    \label{subsec:es6_generators}
      Generator 是什么?

      先来个例子吧。

      \begin{lstlisting}
        function* quips(name) {
          yield "hello " + name + "!";
          yield "i hope you are enjoying the blog posts";
          if (name.startsWith("X")) {
            yield "it's cool how your name starts with X, " + name;
          }
          yield "see you later!";
        }
      \end{lstlisting}

      这是一个"\href{http://people.mozilla.org/~jorendorff/demos/meow.html}{会说话的猫}"小例子的一些代码,这可能是今天互联网上最重要的一种应用。(你可以点击\href{http://people.mozilla.org/~jorendorff/demos/meow.html}{这个链接},玩一玩这只猫。当你困惑的时候,再回到这篇文章寻找解释。)

      看起来像函数,是吗?这个叫做生成器函数(generator-function), 和普通的函数有很多共同点。但你很快就能看出差别:

      \begin{itemize}
        \item 普通的函数以关键字 \keyword{function} 开头,而生成器函数(generator-function)以 \keyword{function*} 开头。
        \item 在 generator-function 中,\keyword{yield} 是关键字,语法上就像 \keyword{return}。不同点是虽然一个函数只能返回一次,但 generator-function 可以 yield 很多次。\keyword{yield} 关键字暂停了 generator 的执行,之后还可以继续往下执行。
      \end{itemize}

      就是这样,这就是普通的函数和 generator-function 最大的不同。普通的函数不能暂停,而 generator-function 就可以。

  \subsection{generators 是怎么执行的}
    \label{subsec:generators_是怎么执行的}

      当你调用 quips() 这个 generator-function 时候发生了什么?

      \begin{lstlisting}
        > var iter = quips("jorendorff");
          [object Generator]
        > iter.next()
          { value: "hello jorendorff!", done: false }
        > iter.next()
          { value: "i hope you are enjoying the blog posts", done: false }
        > iter.next()
          { value: "see you later!", done: false }
        > iter.next()
          { value: undefined, done: true }
      \end{lstlisting}

      你可能对普通的函数比较熟悉,知道他们怎么执行的。当你调用他们的时候,他们就立即执行,一直到他们返回或抛出异常。

      调用一个 generator 形式上看起来一样:\keyword{quips("jorendorrff")}。但当你调用 generator 的时候,他不会立即开始执行。相反,他返回一个 Generator 对象(上个例子中叫做 \keyword{iter})。你可以把这个 Generator 对象看成是函数调用,在 generator-function 执行第一段代码之前就会停住。

      之后你每次调用 Generator 对象的 \keyword{.next()} 方法,函数就会往下执行直到遇到下个 \keyword{yield}。

      这就是为什么每次我们每次调用上面的 \keyword{iter.next()}, 我们就会得到一个不同的字符串值。这些值是由 \keyword{yield} 产生的。

      最后一次 \keyword{iter.next()} 调用,终于执行到 generator-function 的最后,打印的对象的 \keyword{done} 字段由 \keyword{false} 变为 \keyword{true}, 而此时 \keyword{.value} 字段变为 \keyword{undefined}。

      现在是时候回到我们的"说话的猫"这个演示页面。如果把 \keyword{yield} 放进 loop 会怎样?

      在技术术语上,每次 generator yields, 他的栈结构(局部变量,参数,临时值,当前执行到的位置)会从栈中移出。但 Generator 对象保留着栈的引用,这样之后的 \keyword{.next()} 调用就可以重新继续执行。

      值得注意的是 Generators 不是线程。有线程的语言,多块代码可以同时运行,这样会导致资源竞争,不确定性以及性能问题。Generators 和这个不同。generator 执行时,执行是有确定的顺序的,不会并发。不像系统进程,generator 只会在 \keyword{yield}
      标记的地方暂停。

      我们现在知道了 generators 是什么了。我们知道 generator 执行的时候,暂停自己,然后再继续执行。那么问题又来了,这种奇怪的特性有什么呢?

  \subsection{Generators 是迭代器(iterators)}
    \label{subsec:generators_是迭代器_iterators_}
      上周,我们看到迭代器不仅仅是是个内置的类。是可以扩展的。你可以创建自己的迭代器,只需实现两个方法:: \keyword{[Symbol.iterator]()} and \keyword{.next()}。

      但去实现一个接口要花点时间。我们来看看一个 iterator 是如何实现的。先来看个例子,我们来写一个简单的 \keyword{range} iterator ,作用是从一个数字一直计数到另一个数字,类似于 \keyword{for(;;)} 循环。

      \begin{lstlisting}
        // This should "ding" three times
        for (var value of range(0, 3)) {
          alert("Ding! at floor #" + value);
        }
      \end{lstlisting}

      实现的一种方法是使用 ES6 的 class。(如果对 class 语法不是很清楚,别担心,我会在后面的文章讨论)。

      \begin{lstlisting}
        class RangeIterator {
          constructor(start, stop) {
            this.value = start;
            this.stop = stop;
          }

          [Symbol.iterator]() { return this; }

          next() {
            var value = this.value;
            if (value < this.stop) {
              this.value++;
              return {done: false, value: value};
            } else {
              return {done: true, value: undefined};
            }
          }
        }

        // Return a new iterator that counts up from 'start' to 'stop'.
        function range(start, stop) {
          return new RangeIterator(start, stop);
        }
      \end{lstlisting}

      这就是 iterator 的一种实现,类似于 Java 和 Swift 语言。这段代码有 bugs 吗? 不好说。看起来和 for(;;) 循环不像。iterator 的这种特性迫使我们拆开循环。

      现在你可能对 iterators 感到有点不爽了是吧。iterator 挺好用,但不好实现。

      在 JS 语言中引入一个新的令人费解的控制结构让 iterator 更容易实现,这种做法我们或许想不到。但既然我们有 generators, 我们可以使用它吗?试试看吧:

      \begin{lstlisting}
        function* range(start, stop) {
          for (var i = start; i < stop; i++)
            yield i;
        }
      \end{lstlisting}

      上面四行代码就是对上面的 23 行代码的一个简单的替代包括整个 RangeIterator class. 这可能是由于 generators 是 iterator 的缘故。所有的 generators 都有一个内置的 \keyword{.next()} 和 \keyword{[Symbol.iterator]} 。

      不用 generators 去实现 iterators 就像被迫全部用被动语态写一篇长的 email。RangeIterator 这个类的代码又长又奇怪,因为他必须描述一个循环而且还不能使用循环语法。Generators 就是很好的一个替代品。

      我们还可以怎样用 generator 的特性去实现 iterator ?

      \begin{itemize}
        \item 让任意的对象可迭代(iterable). 你就写一个 genertor-function 遍历 this 对象,每次用 yield 产生对应的值。再把 generator-function 安装为 [Symbol.iterator] 这个对象的方法。
        \item 简化构建数组的函数。假设你有个函数,每次被调用时返回一个数组, 就像下面这个:
        \begin{lstlisting}
          // Divide the one-dimensional array 'icons'
          // into arrays of length 'rowLength'.
          function splitIntoRows(icons, rowLength) {
            var rows = [];
            for (var i = 0; i < icons.length; i += rowLength) {
              rows.push(icons.slice(i, i + rowLength));
            }
            return rows;
          }
        \end{lstlisting}
      \end{itemize}

       Generators 可以让这种代码更简洁:

       \begin{lstlisting}
        function* splitIntoRows(icons, rowLength) {
          for (var i = 0; i < icons.length; i += rowLength) {
            yield icons.slice(i, i + rowLength);
          }
        }
       \end{lstlisting}

      这种方式与上面的方式的唯一不同就是,这种方法不会一次性计算所有的结果后返回一个数组,而是返回一个 iterator, 根据需要一个一个去计算。

      \begin{itemize}
        \item 返回不寻常(数组)大小的结果。你不能创建一个无穷的数组。但你可以返回一个 generator 去产生一个无穷的序列,根据你的需要去产生任意多个数据。
        \item 重构复杂的循环。你曾写过又庞大有丑陋的函数吗?你想不想把他分成两个简单的部分?Generators 就是为数不多的工具之一,能帮你重构代码。当你去写一个复杂的循环的时候,你可以把产生数据的部分单独用 generator 去重构。然后把你的循环改成 \keyword{for (var data of myNewGenerator(args)) }.
        \item 作为处理 iterables 的工具。ES 6 没有专门提供一些库,为用来对任意可迭代数据集进行过滤、映射、或 hack,但 generators 仅用几行代码就能实现。
      \end{itemize}

      假设你需要一个类似于 \keyword{Array.prototype.filter} 的方法,用来对一些 DOM 节点进行操作,而不是数组。很简单,就像下面这样:

      \begin{lstlisting}
        function* filter(test, iterable) {
          for (var item of iterable) {
            if (test(item))
              yield item;
          }
        }
      \end{lstlisting}

      generators 是不是很有用?当然,这是一种很简单的方法,能够实现自定义的迭代器,而迭代器是 ES6 中的新标准。

      但 generator 不仅仅能实现上面这些功能,而且这还不是 generator 能做的最重要的事。

  \subsection{generators 和异步代码}
    \label{subsec:generators_和异步代码}
      我曾经写过这样的代码:

      \begin{lstlisting}
                  };
                })
              });
            });
          });
        });
      \end{lstlisting}

      你曾经也可能写过类似的代码。异步 API 返回一个回调函数,这就意味着你每次都要写一个额外的匿名函数。如果你要用代码做三件事,你不是去三行代码,而是像上面那样层层缩进。

      下面是我写的另外一些代码:

      \begin{lstlisting}
        }).on('close', function () {
          done(undefined, undefined);
        }).on('error', function (error) {
          done(error);
        });
      \end{lstlisting}

      异步 API 通常都有错误处理机制,但没有异常处理机制。不同的 API 有不同的惯例。多数情况下,错误就被忽略了。甚至成功的结束了,但还是会被忽略。

      这些问题就是异步编程的代价。我们不得不接受这些比同步代码难看多了的异步代码。

      Genertors 给我们了一些希望。\href{https://github.com/kriskowal/q/tree/v1/examples/async-generators}{Q.sync()} 是一个例子,用 generator 和 promise 去让代码写起了更像同步的代码。

      \begin{lstlisting}
        // Synchronous code to make some noise.
        function makeNoise() {
          shake();
          rattle();
          roll();
        }

        // Asynchronous code to make some noise.
        // Returns a Promise object that becomes resolved
        // when we're done making noise.
        function makeNoise_async() {
          return Q.async(function* () {
            yield shake_async();
            yield rattle_async();
            yield roll_async();
          });
        }
      \end{lstlisting}

      主要的不同就是异步的写法必须在每个调用异步函数的地方加上 \keyword{yield} 。

      所以 generators 让我们以更习惯的方式去进行异步编程。这项工作会继续进行下去,因为 ES7 正在已经在为这个努力了,灵感来自 C\#.

   \subsection{我什么时候可以使用 generator}
     \label{subsec:我什么时候可以使用_generator}
        在服务器端,如果你采用了 io.js 框架,你可以使用 ES6 的 generator 。或者如果你使用 nodejs 的话,使用时在命令行加上 --harmony 选项。

        浏览器中,目前只有 Firefox 27+ 和 Chrome 39+ 支持 ES6 的 generator。其他版本的浏览器你可能需要 Babel 或 Traceur 把你的 ES6 代码转换成 ES5。

        Generators 是由 Brendan Eich 第一p次引入的;他的设计灵感来自于 Python generators , 而 Python generators 是受 Icon 启发的。标准化的道路是曲折的,语法和行为都在变化。ES6 的 generator 是由 Andy Wingo 集成到 Firefox 和 Chrome 中的,Bloomberg 赞助了他。
